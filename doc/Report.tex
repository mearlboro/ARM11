\documentclass{article}
\usepackage{graphicx}
\usepackage[english]{babel}

%%%%%%%%%% Start TeXmacs macros
\newcommand{\tmrsup}[1]{\textsuperscript{#1}}
%%%%%%%%%% End TeXmacs macros

\begin{document}

\title{Final Report}\author{Lorenzo Calegari, Ioana Radu, Madalina Sas, Tudor
Cosmiuc}\maketitle

\section{Extension}

\subsection{Dotsies visual tool}

\ \ \ \ Dotsies (http://dotsies.org) is a font that uses dots instead of
letters. In Dotsies, each letter of the latin alphabet can be represented as a
one-colum array of 5 squares, or dots, which can either be on or off (black or
white). It is clear how each letter can in turn be mapped to a 5-bit binary
number; we can therefore represent up to 2\tmrsup{5}=32 values in this
fashion. Punctuation characters and numbers are not included.

For our extention we plan create a visual tool to aid the learning of Dotsies.

We will position a column of five red LEDs on the breadboard provided. This
will be connected to the Raspberry Pi which in turn will be interfaced with a
keyboard to collect input from the user. The program will display a single
letter at random using the LEDs array to represent it in a Dotsies fashion.
The program will then wait for the user to press a single key which they think
corresponds to the Dotsies character produced by the LED strip. There are two
addional red and green LEDs positioned across the breadboard; these will
provide feedback to the user and signal whether they guessed the character
correctly. The user will be continuosly asked to guess new letters until they
decide to terminate the program.

For our extension we will need to interface the Raspberry Pi with a keyboard
device to collect input from the user. This might involve the writing of
drivers , although there is currently one open-source library that, with a bit
of fiddling can do that for us. More information at:
http://www.cl.cam.ac.uk/projects/raspberrypi/tutorials/os/input01.html

\section{Group Reflection}

\subsection{Group communication}

\ \ \ \ Communicating in a group can be challenging because of many factors.
Some of these factors are the availability of your team mates and timing when
trying to organise the team. We often found that we couldn't synchronise our
working times and we had difficulties setting up a time to meet. However, we
overcome these problems by always staying in contact with one another and by
managing to improve our individual working schedules for the benefit of the
team. By the end of the project we feel that we will work like a well oiled
machine.

Another problem when trying to communicate in a team is effectively expressing
one's ideas. We realised that in a team you have to be considerate of your
colleagues, but also convey your opinions in a compelling way. In this aspect,
everyone in our team managed to freely speak their minds and create an
environment in which we were able to do efficient brainstorming.

\subsection{Work distribution}

\ \ \ \ In the beginning, splitting the work load was quite difficult because
of the nature of the project and our inexperience with working in groups. The
fact that the project didn't really have a skeleton or detailed directions
about how we could implement the project took us aback as we tried to organise
ourselves. It was Lorenzo that finally took the leader position and wrote the
skeleton files and divided the project into smaller tasks that each of us
could implement separately.

Another obstacle we had to face was figuring out the timeline of out project.
In particular, coming up with a plan of attack : what tasks needed to be
implemented first, how the tasks related to one another, on what assignments
could we work on at the same time and the dependecies between them.

We usually worked on whatever parts we wanted and tried to give to each of us
tasks that we would enjoy working on. We feel that this approach helped with
keeping the enthusiasm alive for this project, as we only worked on the parts
that we were interested in.

We experimented many approaches for working. Sometimes we would stay up at
night in the weekends for a late programming session. Another one of the
approaches we tried was working in pairs. Working in pairs was close to a
success. Having someone in your back or backing up someone has reduced the
possibility of mistake and, furthermore, minimized the debugging time. The
issue with this was that somehow the pairs didn't change much, and most of the
actual writing was done on only two branches. Another observation would be
that after a while of working together, both pairs started having very similar
ideas for doing the same things without actually talking to eachother. We
think this is the biggest achievement we got as programmers for now: being able
to adapt to other people's style and, moreover, ways of thinking.

\subsection{Future improvements}

\ \ In the future we think that we could improve our group dynamics by making
a few changes in how we work.

Firstly, we would like to revise the way we managed our time by dividing it
and assigning set periods of time for each smaller section of our program.
Furthermore, we could assign smaller deadlines for each part and create group
meetings to discuss our progress.

Secondly, we believe that we could better the way we organise our projects by
having a couple of preliminary meetings in which we will extensively discuss
and commonly agree upon the design of our project and the way we want to
implement this design.

Consequently to having set up a plan of action for our project, we feel that
it would benefit our team to distribute from the beginning each part of the
project between us, as it would make it much easier in the future for each one
of us to know precisely what we have to do.

In conclusion, we think that by coordinating ourselves in a better way and by
setting a few rules that we have to obide by we could make great improvements
to how our team functions in the future.

\section{Individual reflections}

\subsection{Tudor Cosmiuc's reflection}

\ \ \ \ Although we got slightly behind on time, I believe the group managed
to work well together and split work effectively. However, I would have
preffered to have had more communication during the early stages of the
project so that we could all contribute to the general look of the code. Other
than that , it seems that this group showed that we could work well in a team
and overcome stressful situations without much hassle.

I enjoyed both writing new code as well as debugging old one, but I hope to
make more of a difference next time when it comes to the structure of the code
as well. Although we met oftenly, the different working schedules were
obvious, and therefore it was difficult for all of us to work together for any
extended period of time. I do hope that in future projects we would be able to
spend more time working as a group instead of working separately due to
different prefferences of working hours.

\subsection{Ioana Radu's reflection}

\ \ \ \

I feel that we make a good team, we work well together but we also have
different personalities and we are different individuals. What I think I
brought to the team is the fact that I always tried to reposition the team on
the most realistical path. I believe that I helped the team be down to earth
and not adventure onto dangerous and too ambitious paths.

What I would have changed is the way I distributed my energy and my drive for
work. In the beginning I was very eager to work on the project, but that soon
faded away after we finished with our emulator. I think that me not having a
constant desire to work on the project might have affected the way we managed
our time and the fact that we found ourselves a bit behind with the assemble
part of the project.

I hope that in the future I will be more organised and efficient in my time
and energy manangement and that our team will grow stronger with every
project.

\subsection{Madalina Sas's reflection}

\ \ I believe that working for this project has taught us many things. Since
we are quite close as friends, working for this project was 'comfortable'. I
believe there has been a strong improvement compared to how we worked on the
topics project. We met 3-4 times weekly in the labs to work, subsequently
having parts of it done quite quickly (the emulator was done in 3 days, but
then optimised).

After the checkpoint with Maria, we tried to work on separate branches and
merge afterwards: I am not sure if this method worked perfectly, because we
ended up having code which has been written twice. Moreover, merging was very
complicated in the end. Still, there has been a very good level of
communication between us, and most of the duplicate code was written in full
awareness of all the team members. I assume this happened because of our
passion for writing C and I also think it sometimes helped finding the optimal
way of doing something.

Last, but not least, we might have been slightly disorganised. Discussions
regarding the project were oftenly blended into our normal talks. I think we
should also try the opposite: talk about the project and specifically assign
work in strictly project-related meetings.

\subsection{Lorenzo Calegari's reflection}

\ \ Working on this project has been quite an experience indeed. Due to the
nature of my persona and, most importantly, my newly-discovered love for C, I
really got carried away and was soon fully absorbed by it. I have been
extremely keen on performing well and eager to achieve execellent results both
for myself and my team.

Arguably, I felt that the best way to do so was to go away and work on my own,
taking up one task after the other and trying to produce and contribute as
much and as best as I could. While I believe this was essential for the
completion of the project by deadline, I also felt that I left my team mates
with not much to do or not many ways to help. As the team leader, I could have
split the workload more efficiently and organized and assigned tasks with more
regularity since the beginning of the project. I believe I haven't allowed by
team mates to express their full potential and show what they are capable of.

We are a tidy lot but we have just begun to really know each other. We are all
very proud programmers and that makes it really hard to convey ideas, opinions
and judgments without a risk for misunderstandings. Overall I believe we did
really well and I do look forward to growing together as a team and as
individuals.

\end{document}
